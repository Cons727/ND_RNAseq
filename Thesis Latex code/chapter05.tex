\chapter{Discussion}
\label{chapter:Discussion}

In this research seven analyses were performed on 13 diverse public databases of RNA-seq expression data with AD, PD, or HD from distinct sections of the human brain. The main objective is to decipher the alterations which drive the pathogenesis of each disease, their differences and similarities by interrogating the transcriptome. The results presented in Chapter \ref{chapter:Results} are given by analysis, in this chapter the results are incorporated in order to provide a thorough explanation.

PD is a chronic progressive disease which starts with a noticeable tremor and it develops until the automatic movements are lost and the muscles become rigid; this ND affects mainly the motor system by the reduction of neural cells which prompts a decrease of dopamine. Although, the cause of neural death is not well defined, the principal characteristics of PD are agglomerations of misfolded $\alpha$-synuclein into Lewy bodies. Moreover, in recent years, studies have concluded that neuroinflammation is also involved in the pathogenesis of PD \cite{hirsch}. 

According to the interpretation of identification of immune cell types proportion on PCx-PD, the increased abundance of neutrophils, monocytes and macrophages suggests inflammation in the tissue. Furthermore, ORA resulted in pathways related to the immune system response and secretion; both routes are involved in inflammation. GSEA gave pathways associated with brain (neuron synapse) and, more interestingly, protein localization, tau protein, and substantia nigra development pathways were enriched. Lewy bodies are comprised of $\alpha$-synuclein and tau proteins \cite{irwin}. In addition, the neural death starts in the substantia nigra, a midpart of the brain. Therefore, the mentioned pathways are coherent with the pathophysiology of the disease. Additionally, the main actors of neuroinflammation are microglia and astrocytes, since the former are the macrophages of the brain, and the latter are crucial regulators of the immune response in this site \cite{colombo, leng}; then is acceptable to have results suggesting a higher proportion of microglia and astrocytes in PD than in the control samples, as well as a reduction in neuron abundance given the cell death common in this condition. What is more, the inflammation occurring in PD, comprises the blood-brain barrier (BBB), allowing bigger molecules and cells to enter the brain. Finally, it seems that PD may be linked to an increase CN in chromosome 17 and 19.

When analyzing the results for the molecular subtypes of PD, a similar explanation can be suggested. The three subtypes had enriched pathways related to inflammation, had an increment of microglia and a reduction in neuron proportion in PD samples. However, S1 and S3 were enriched in routes associated with vascular development, and S3 with extracellular matrix processes. Furthermore, S2 had more pathways related to the brain, including glia cell differentiation, gliogenesis, and ensheatment processes; ORA and GSEA resulted in oligodendrocyte development, which is coherent with the increment of this cell given in the identification of brain cell types proportion results. Therefore, it seems that S1 is characterized by some type of vascular development in PD, S3 is implicated in vascular development and in extracellular matrix process, while S2 has an increase in oligodendrocyte abundance and active ensheatment process. The latter may suggest that PD-S2 has fracture myelin or demyelination which enhances the proliferation of oligodendrocytes, the producers of myelin in the brain. Lastly, it appears that the CNV between subtypes and control are highly contrasted in almost all the chromosomes of the human genome.

The comparison among GWAS and the DEGs of this study gave interesting findings, specially in the variants not presented in GWAS (unnanotated variants). CEBPD gene which is involved in the regulation of immune and inflammatory responses, is related to developmental coordination disorder, and is up-regulated in PD in this study; that being so, the over-expression of CEBPD may be one of the causes of neuroinflammation. EGR2 plays a role in the regulation of myelin, is related to brain development, and it has been associated with neuromuscular disease; in this study, this transcription factor is down-regulated giving a possible explanation to the oligodendrocyte proliferation of S2. In addition, the common variants mapped by GWAS, matched with MMRN1, a gene which may have functions in extracellular matrix and adhesion; MMRN1 is up-regulated in this research and may be a plausible justification to the enriched pathways of S3. Finally, ITPKB regulates the level of inositols, the alteration in the molecule concentration may be altered in brain disorders \cite{brand}. 

As context, HD is a chronic rare disease which causes the progressive degeneration of nerve cells mainly in striatum and cerebral cortex. This disorder causes involuntary movements, called chorea, severe emotional disturbances (depression) and cognitive decline. HD is an autosomal dominant disorder caused by a mutation in the huntingtin gene, a trinucleotide repeat expansion of CAG; the mutant protein has a pathological elongation of polyglutamine tract which gradually damages the brain cells. The number of CAG repeats determines the clinical symptoms, the age of onset, and the age of death. Moreover, huntingtin has been associated to cell membrane, regulation of autophagy, synaptic transmission, formation of the nervous system, control of the neuron survival, and cell adhesion. The mutant protein aggregations as inclusion bodies in neurons are formed by the proteolysis of the plyglutamine fragments of the mutant huntingtin. Until today, selective proteolysis and nuclear translocation of the polyglutamine fragments are considered the first steps of pathogenesis. The known mechanisms of HD are related to transcription dysregulation, mitochondria and energy metabolism dysfunctions, axonal transport defects, neuroinflammation, oxidative stress, and loss of function of huntingtin \cite{illarioshkin}.

According to information given above, HD has been associated with inflammation, agglomeration of the polyglutamine fragments in the neural cell, and energy metabolism disorder. For PCx-HD, ORA and GSEA resulted in pathways related to immune response and inflammation, cellular respiration, protein localization, and main axon; all the routes are linked to the mechanisms mentioned in the review. Moreover, the tau protein pathway was also enriched in this dataset, which may suggest a similar process of protein translocation to the nucleus as in AD. In addition, the interpretation of cell proportion suggested the presence of inflammation and active immune response, as well as an increment in microglia and astrocytes in HD; both related to a possible neuroinflammation. Furthermore, neuron abundance was lower in HD than in control samples, implying a process of neuron death. An interesting result is that the node of Ranvier and substantia nigra development pathways were also enriched; recent studies have suggested significant decrease in neurons in substantia nigra leading to a reduction in dopamine release which may cause the chorea \cite{cepeda}. In addition, the CNV plot suggests no significant change.

The interpretations of peripheral blood samples with HD were split between female and male given the significance of the sex covariate. The female samples gave similar result in immune cell proportions as prefrontal cortex, which adds up to theory since immune cells travel by blood. Nevertheless, the brain cell type proportions were not concordant to PCx-HD which might be caused by the BBB. Additionally, neuroinflammation was also associated in Blood-HD-f according to GSEA results where granule and immune processes were enriched. What is more, Blood-HD-f and its three molecular subtypes output a low proportion of resting mast cells and no significant change in active mast cells; mast cells are the first responders to injury in the brain, have influence in astrocytes, microglia and neurons, and they may be able to let through the BBB toxins and other immune cells which exacerbate the inflammation (the BBB becomes permeable during neuroinflammation). Hence, a decrease of resting mast cells may lead to a permeable BBB. In addition, the results of this analysis suggested inflammation for S3. Additionally, S2 and S3 were enriched in immune system routes, whereas S1 and S3 were enriched in wound healing and cellular matrix pathways. Interestingly, S1 was related to cell adhesion, S2 was linked to proteolysis and cell killing, and S3 with splicing.

On the other hand, the male samples and its subtype 2 implied an inactive immune response due to the increase of resting memory CD4 T cells in HD. Nonetheless, astrocyte abundance appeared lower in case than in control. Additionally, GSEA resulted in pathways related to immune system, granule and cell membrane pathways, while ORA output routes about telomeres and chromosome organization. What is more, S2 was associated with glia activation, neuroinflammation and immune system pathways; what is interesting is that glia activation may lead to the activation of the immune system which drives to neurinflammation. Another interesting result is that S1 and S2 were enriched in pathways related to autophagy, and splicing; S1 was also related to catabolic processes and ubiquitination. As mentioned above, huntingtin is hypothezied to have functions in the regulation of autophagy, cell adhesion, and the agglomeration of ployglutamine have shown to comprise ubiquitin as well \cite{illarioshkin}. Additionally, the CNV analysis of the female set resulted with chromosome 17, 19 and 20 with high CN and low in chromosome 5, while the male set has the inverted direction.

Since HD has few GWAS variants, as seen in Table \ref{tab:gwas}, just three SNPs matched with the DEGs results. In prefrontal cortex, MSX1, from the unannotated motifs list, appeared over-expressed in HD. MSX1 acts in the cranofacial development, and is related with neuroscience route and PD. Moreover, peripheral blood from male samples, matched with one mapped variant: PTPRM, a gene involved in cell-cell adhesion, and cell growth. NR1H2 gene, from unannotated motifs list, was identified as down-regulated in male peripheral blood samples; this gene modulates gene activation and is expressed in blood mononuclear cells. Since NR1H2 has not been mentioned in GWAS, this gene could be suggest as a novel biomarker for HD using blood samples, although according to the Venn diagrams of tissue comparisons, peripheral blood is not able to show all the changes occurring in the prefrontal cortex. However, the six genes which were common among PCx-HD and Blood-HD-m have interesting implications: CXCL1 (role in inflammation), CPVL (role in proteolysis), IL1BR1 (cytokine receptor of interleukin 18), STX11 (syntaxin with role in intracellular transport), ANG (mediator of new blood vessels), and EMS1 (endothelial-leukocyte interactions).

The comparison of DEGs made between PD and HD in prefrontal cortex, may imply a common pathogenesis. The shared pathways were about plasma membrane components, synaptic signaling, G protein coupled receptor signaling, immune system processes, and response to dopamine. The last path may be reason of both diseases presenting chorea-like symptoms, and the penultimate route is coherent with the aforementioned mechanism of inflammation. 

AD is a progressive disease, the most common cause of dementia, and the source of the disorder is not well understood. Some hypotheses are related to genetic factors (APOE), other to abnormal accumulation of proteins in the extracellular space (amyloid-$\beta$ plaques) and in the intracellular space (tau protein neurofibrillary tangles), and to dysfunction of olygodendrocytes; all of these drives to an advancing loss of brain function. The malformed tau proteins enter neurons and disintegrate microtubules which collapses the cell structure and its transport system, leading to cell death \cite{iqbal}. Amyloid-$\beta$ plaques are thought to be cytotoxic to neurons by disrupting the calcium ion homeostasis in the cell which leads to apoptosis \cite{yankner}. Moreover, this plaques have been linked to demyelination and profound loss of olygodendrocytes, which causes more amyloid production and phosphorylation of the tau protein \cite{mitew}. Also, inflammation by microglia has been suggested as key pathogenesis of AD \cite{leng}. As mentioned before, tau proteins have been shown to be present in Lewy bodies, and Lewy bodies are normally present in AD patients \cite{kotzbauer}.

Summarizing the results of AD obtained in this study, all the tissues were enriched with brain processes pathways, such as neurotransmitters, synapse, and ion channels; some tissues were associated with substantia nigra development. Interestingly, four out of five tissues were enriched with electron transport chain pathways. According to \cite{eckert}, the plaques and tangles activates mitochondrial dysfunction by oxidative phosphorylation, reactive oxygen species production, alterations in mitochondria dynamics and proteins leading to variation in the electron transport chain mechanism. Furthermore, AD has been associated with substantia nigra pathology, such as $\alpha$-synuclein accumulations, phosphorylated tau aggregations, and neuron loss \cite{burns}. Also, the majority of the datasets suggested inflammation through the interpretation of the immune cell types analysis, but no pathway was enriched in this subject. In addition, all datasets resulted in a reduction of neurons and an increase in astrocyte population (not in fusiform gyrus); in just frontal lobe and fusiform gyrus an increment of microglia was found. What is more, all the tissues coincide in a positive CNV
at chromosomes 16 to 19 and negative in X chromosome.

In specific case, the samples of the parietal tissue, both female and male, were enriched with electron transport chain, node of Ranvier and substantia nigra development, alongside the normal brain pathways mentioned above. Moreover, the male set had over represented genes associated with cognition and memory, which are hallmarks of AD. Both sets have increased astrocyte and reduction in neuron abundance. Additionally, the female samples suggests inflammation given the results of the cell proportion identification. Moving on to the temporal tissue, both sexes were enriched with normal brain pathways and cellular respiration, as well. What is more, the union of both sets were related with chronic axonal neuropathy, which may be associated with AD. Additionally, astrocytes are increased in AD samples, while neurons are decreased in both sexes.

Interesting results were obtained in the hippocampus tissue, where both sexes were enriched with cellular respiration and substantia nigra development; the female samples resulted in pathways associated with memory, and the male counterpart was related to inclusion body assembly and long-term memory. It is known that tau protein forms aggregates as well as amyloid-$\alpha$ in AD. Also, AD affects short-term memory and long-term memory as the disease progresses. Interestingly, one specific gene common among female and male sets, which is up-regulated in AD, is found in paraneoplastic cerebellar degeneration: CDR1. In contrast to other datasets, hippocampus does not imply inflammation neither by the results of immune cell abundance nor of the functional analysis. Continuing with frontal tissue, the male samples were enriched with oxidative stress neuron death and myelin ensheatment pathways; both routes are key in AD as mentioned in the previous given information. However, recalling preceding ideas, female and male set did not converge in any common DEGs, CNV or brain cell types; the male results appeared to be in opposite direction compared to female.

Lastly, the over represented pathways of fusiform gyrus are related to synapse and ion channels, same as other datasets. Both female and male samples were enriched with cellular respiration. Nevertheless, female set was the only dataset associated to tau protein route, which seems pivotal in AD pathogenesis. Additionally, both sexes may imply inflammation by the immune cell proportions and the shared pathway of defense response. Also, their subtypes were enriched with immune system process and inflammation response routes and concurring with the result of immune cell types abundance. Furthermore, female, male and their subtypes (except Fus-AD-m-S2) have increased microglia and reduced neuron proportions. As mentioned, all the subtypes were enriched with synapse, ion channel routes, inflammation, and vascular development. However, Fus-AD-m-S2 was associated with cognition, memory and locomotion activity; the first two pathways are clearly related to AD. 

In contrast to PD and HD, in GWAS database several variants are associated with AD. Specifically, 22 DEGs were matched with 29 common originally mapped SNPs, and 2 unannotated motif variants with 2 genes only in the fusifurm gyrus tissue, and four distinct DEGs were paired in frontal tissue. The novel affected genes by variants mentioned in this report are PGR, FOXJ1, RREB1, SP100, and REST.

Given the interpretation for AD, PD and HD, it can be seen that the three disease have common pathophysiologies, such as neuroinflammation, agglomeration of proteins in the brain (either in intracellular or in the extracellular space), substantia nigra pathologies, and neural death. According to the Venn diagram in Fig. \ref{fig:common-tissues-down}, \ref{fig:common-tissues-up} of similar tissues, AD and HD share an increased in inflammatory and defense response, cell differentiation, and epithelium development, and a decreased process of synapse. Additionally, AD and PD have common over-expressed cell activation and immune system response, and under-expressed synapse, axon, and neuron projection. AD, PD and HD have up-regulated immune and inflammatory processes, and apoptotic cell death, while synapse is also down-regulated in the three diseases.

Furthermore, the repurposed drug suggested in this study aims the pathologies of each disease, for example flurbiprofen targets inflammation of PD, and simvastatin is hypothesized to have neuroprotective properties useful for treatment of PD. For HD, many of the resulted drugs aims to protect from oxidative stress neural death, also amitriptyline enhances mitochondrial function, benzamil enhances the degradation of polyglutamine aggregates, and remoxipride targets the dopaminergic receptors. Finally, drugs referred here to be possible treatment of AD are also claimed to have neuroprotective characteristics, anti-inflammatory effects, and oxidative stress protection. More specifically, some drugs target amyloid-$\beta$ plaques, or tau proteins tangles. Interestingly, acetyl salicylsalicylic acid was obtained in the majority of the analyses. 

As conclusion, this study leveraged seven analyses in order to have a complete interpretation of AD, PD, and HD using public RNA-seq expression data. The methodology consisted in a preprocessing and an analytic phases. Preprocessing steps of filtering, normalization, batch correction and covariate identification were required before the implementation of the analytical algorithms. The analytic phase comprised of molecular subtyping, brain and immune cell proportion estimation, differential expression analysis, funtional analysis, copy number variation, drug repurposing and finishing with the comparison of GWAS. As implied, several algorithms were employed in order to fulfill the mentioned pipeline and attain the objectives, including \textit{limma}, \verb|ComBat|, NMF, \textit{BRETIGEA}, CIBERSORTx, among others. 

Using public datasets conveys certain limitations, such as distinct normalization of the data (FPKM, RPKM, counts), different tissue of origin (specific parts of the brain), and, more consequential, insufficiency of clinical metadata. Nevertheless, this limitations were surpassed in the extent this research was accomplished. Future work could be done using the proportion of immune and brain cell types as covariates for DGE analysis and molecular subtype identification. Moreover, gathering the raw FASTQ files of these datasets for additional the analyses, such as alternative splicing, and reprocessing the data to have standardized expression matrices could lead to novel findings.

In this research, a possible biomarker for HD was suggested, as well as plausible treatments for the studied NDs. AD, PD and HD have common pathologic mechanisms which lead to neural death, including neuroinflammation, protein agglomerations, and immune response of microglia and astrocytes. Additionally, neural death drives a decrease in synapses and neuron projection. The functional analysis suggested a common dysfunction in mitochondria and in myelin ensheatment. Important DEGs were found for each disease and those genes were compared with GWAS and unannotated variations. Although NDs are complex diseases, transcriptome analyses can bring us closer to their understanding.